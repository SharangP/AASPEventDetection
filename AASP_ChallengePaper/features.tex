In order to classify events, our system extracts an extensive set of features from the provided auditory data. This set includes frequency-based features as well as time-based features, which, together, constituted a 39 element feature vector. Each of the features were computed in 50ms windows to balance frequency and time resolution.

\subsection{Frequency-Based Features}
The set of frequency-based features includes the spectral centroid, spectral flux, spectral entropy, short time energy (STE), spectral roll-off, and Mel-frequency cepstral coefficients (MFCCs). Each of these feature highlights different spectral properties of the signal. Short time energy is analogous to the volume of the event and the spectral centroid is essentially the center of mass of the spectrum. They are both expected to be reliable indicators of
silence. The spectral flux, also called spectral variation, measures how quickly the power spectrum changes. It can be used to determine the timbre of the audio signal. Finally, the MFCCs, some of the most widely used acoustic features, measure the power spectrum of the signal in one of 20 computed bands equally spaced on the Mel frequency scale. Each of these features were computed with either hamming or rectangular windows. In addition, the mean and standard
deviation of each frequency-based feature, excluding MFCCs, was taken across entire training events.

\subsection{Time-Based Features}
Time-based features consisted of loudness, wavelet decomposition coefficients, and autocorrelation. Loudness was measured simply as the rms value of a windowed signal, and provided incremental but significant new information to the system. A three layer wavelet decomposition was implemented using the Discrete Meyer wavelet, and the mean and standard deviation of the wavelet coefficients were computed. The autocorrelation proved to provide the maximum information of any
of the time-based features, improving performance by over 15\%. The mean, median, and standard deviation of each windowed autocorrelation was taken.
