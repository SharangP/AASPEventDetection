In order to classify events, our system extracts an extensive set of features from the provided auditory data. This set includes frequency-based features as well as time-based features, which, together, characterize the data in relative detail. Each of the features were computed in 50ms windows to balance frequency and time resolution.

\subsection{Frequency-Based Features}
The set of frequency-based features includes the spectral centroid, spectral flux, spectral entropy, short time energy (STE), spectral roll-off, and Mel-frequency cepstral coefficients (MFCCs).
Each of these feature highlights different spectral properties of the signal.  Short time energy is analogous to the volume of the event and the spectral centroid is essentially the center of mass of the spectrum. They are both expected to be reliable indicators of silence. The spectral flux, also called spectral variation, measures how quickly the power spectrum changes. It can be used to determine the timbre of the audio signal. Different permutations of these features will be tested and the best combination will be used for the system training and classification stages of our system. 

\subsection{Time-Based Features}
