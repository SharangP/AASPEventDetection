Before events within a continuous input signal can be classified, the segments of the input which contain events must be extracted. A speech segmenter[[INSERT DUDE HERE]] accomplishes this by examining the short time energy and spectral centroid features.  The segmenter examines peaks of the features and checks them against a threshold to determine the onset and offset of events. The onset and offset times are then used to extract the portions of the input signal which contain an event and pass them on to the classification stage. 

A continuous sound file is input into the segmenter which  applies a Chebyshev Type II low pass filter. This step was added to the original segmenter to reduce the amount of background noise and increase onset and offset detection. The segmentation process requires three parameters for feature extraction: window length, step time, and the weighting facter that will be used to compare the signal to a threshold. The segmenter was exhaustively tested against the development data to determine the parameters that provided optimal segmentation. Each run of the segmentation test was checked for the number of events it detected. The runs were narrowed down to those that returned the same number of events as the development data indicated. Then the mean square error was calcuated using onset and offset times and used to determine the optimal choice of parameters.The following is the cost function used to measure the segmenter, where $\tau_{(on)i}$ is the estimated onset time and $tau_{(off)i}$ is the estimated offset time and $t_{on}$ and $t_{off}$ are the truth values.
\begin{align}
{MSE} &=\min{ (\tau_{(on)i}-t_{on} )^2 + (\tau_{(off)i}-t_{off})^2  }
\end{align}
The values for the window length, step time, and threshold weight that results in the minimum MSE were 0.05s, 0.04s and 3 respectively. During the exhaustive process, it was also determined that the threshold weight had to be an integer value. 

The segmenting function returns a two column vector where the first column indicates each onset, and the second column indicates each event offset. The signal between the onset and offset times is used in the classification stage of the system. 