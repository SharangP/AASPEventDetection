% --------------------------------------------------------------------------
% Template for AASP Challenge extended abstracts; 
% to be used with: aasp.sty  - LaTeX style file, and
%          IEEEbib.bst - IEEE bibliography style file.
%
% --------------------------------------------------------------------------

\documentclass{article}
\usepackage{aasp,amsmath,graphicx,url,times,epstopdf,cite}
%\usepackage{aasp,amssymb,amsmath,graphicx,times,url}

% Example definitions.
% --------------------
\def\defeqn{\stackrel{\triangle}{=}}
\newcommand{\symvec}[1]{{\mbox{\boldmath $#1$}}}
\newcommand{\symmat}[1]{{\mbox{\boldmath $#1$}}}

% Title.
% --------------------
\title{Event Detection and Classification}

% Single addresses (uncomment and modify for single-address case).
% --------------------
\name{Sameer Chauhan, Sharang Phadke, Christian Sherland\thanks{Thanks to the Cooper Union}}
\address{Author Affiliation(s)}

% For example:
% ------------
\address{Cooper Union for the Advancement of Science and Art\\
	Electrical Engineering\\
	41 Cooper Square\\
	New York, NY 10003}

% Two addresses
% --------------------
%\twoauthors
%  {John Doe\sthanks{Thanks to ABC agency for funding.}}
%    {ABC University\\
%     500 Rainy Street,\\
%     WC1 4AB London, UK \\
%     johndoe@abc.ac.uk}
%  {Jane Doe\sthanks{Thanks to XYZ agency for funding.}}
%    {XYZ University \\
%     3040 Westfield Avenue \\
%     New Paltz, NY, USA \\
%     janedoe@xyz.edu}

\begin{document}
\ninept
\maketitle

\begin{sloppy}

\begin{abstract}
The IEEE AASP Challenge addresses the problem of acoustic event detection and classification in an office environment. Our system performs segmentation and event classification on a continuous stream of acoustic activity in an office using basic feature extraction techniques and a single layer frame-by-frame classifier. We achieve high classification accuracy in noiseless environments, but performance severely deteriorates in noisy environments.

\end{abstract}

\begin{keywords}
onset, offset, MFCCs, LRT
\end{keywords}

\section{Introduction}
\label{sec:intro}
The task of event detection and classification is fundamentally important in computational auditory scene analysis (CASA). Detecting events such as words in a sentence of speech, the arrival of a bus or train, or any other common occurrence in an audio stream is the first stage in many speech and audio processing systems. Many such systems then need to classify the detected events in order to take appropriate actions.

In this project submission, the problem of event detection and classification, as applied to an office environment, was approached from a pattern recognition perspective. Our system consists of a segmentation stage, a feature extraction stage, and two classification stages. The first stage detects the onset and offset of events in a live recording, the second extracts features from each event that can be used to classify events, and the final stages classify each detected event
on a frame-by-frame basis using a pre-trained likelihood ratio test (LRT) classifier with a MAP decision rule. The performance of the system is evaluated on three datasets: a 5-fold cross-validation using the training dataset itself, the development set with ``perfect'' segmentation, and the raw development set.


\section{Feature extraction}
\label{sec:feature}
In order to classify events, our system extracts an extensive set of features from the provided auditory data. This set includes frequency-based features as well as time-based features, which, together, constituted a 39 element feature vector. Each of the features were computed in 50ms windows to balance frequency and time resolution.

\subsection{Frequency-Based Features}
The set of frequency-based features includes the spectral centroid, spectral flux, spectral entropy, short time energy (STE), spectral roll-off, and Mel-frequency cepstral coefficients (MFCCs). Each of these feature highlights different spectral properties of the signal. Short time energy is analogous to the volume of the event and the spectral centroid is essentially the center of mass of the spectrum. They are both expected to be reliable indicators of
silence. The spectral flux, also called spectral variation, measures how quickly the power spectrum changes. It can be used to determine the timbre of the audio signal. Finally, the MFCCs, some of the most widely used acoustic features, measure the power spectrum of the signal in one of 20 computed bands equally spaced on the Mel frequency scale. Each of these features were computed with either hamming or rectangular windows. In addition, the mean and standard
deviation of each frequency-based feature, excluding MFCCs, was taken across entire training events.

\subsection{Time-Based Features}
Time-based features consisted of loudness, wavelet decomposition coefficients, and autocorrelation. Loudness was measured simply as the rms value of a windowed signal, and provided incremental but significant new information to the system. A three layer wavelet decomposition was implemented using the Discrete Meyer wavelet, and the mean and standard deviation of the wavelet coefficients were computed. The autocorrelation proved to provide the maximum information of any
of the time-based features, improving performance by over 15\%. The mean, median, and standard deviation of each windowed autocorrelation was taken.


\section{Training}
\label{sec:training}
The classification stage of our system employs two levels of classification. As input
the classifier takes in an event signal and first classifies what type of event it is.
Then based upon the type of event the signal is passed through a type-specific 
classifier which determines the exact event contained in the signal. In the second stage
each type has its own classifier.

Each level of classification in the system was trained with features extracted from the 
provided training data. The first stage classifier was trained using the full set of 
extracted features for all events. The second stage classifiers were each trained using 
features extracted only from training signals of the correct type. Additionally, the 
features used for training were chosen to best describe the type of event being classified.


\section{Segmentation}
\label{sec:segmentation}
Before events within a continuous input signal can be classified, the segments of the input which contain events must be extracted. A speech segmenter[[INSERT DUDE HERE]] accomplishes this by examining the short time energy and spectral centroid features.  The segmenter examines peaks of the features and checks them against a threshold to determine the onset and offset of events. The onset and offset times are then used to extract the portions of the input signal which contain an event and pass them on to the classification stage. 

A continuous sound file is input into the segmenter which  applies a Chebyshev Type II low pass filter. This step was added to the original segmenter to reduce the amount of background noise and increase onset and offset detection. The segmentation process requires three parameters for feature extraction: window length, step time, and the weight that will be 0 to compare the signal to a threshold. The segmenter was exhaustively tested against the development data to determine the parameters that provided optimal segmentation. Each run of the segmentation test was checked for the number of events it detected. The runs were narrowed down to those that returned the same number of events as the development data indicated. Then the mean square error was calcuated using onset and offset times and used to determine the optimal choice of parameters.$\tau_{(on)i}$ is the estimated onset time and $tau_{(off)i}$ is the estimated offset time determined by the segmenter, which is checked against the truth values $t_{on}$ and $t_{off}$.
\begin{align}
{MSE} &=\min{ (\tau_{(on)i}-t_{on} )^2 + (\tau_{(off)i}-t_{off})^2  }
\end{align}
The values for the window length, step time, and threshold weight that results in the minimum MSE were 0.05s, 0.04s and 3 respectively. During the exhaustive process, it was also determined that the threshold weight had to be an integer value. 

The segmenting function returns a two column vector where the first column indicates each onset, and the second column indicates each event offset. The signal between the onset and offset times is used in the classification stage of the system. 

\section{Classification}
\label{sec:classification}
After a set of event signals has been segmented, the classifier extracts a full set of 
features for each event signal. For each event, these features are passed into the first
level classifier. Based upon the output of the first stage classifier, each event is 
labeled with its subgroup.

Each event signal was then passed through the classifier that corresponds to its subgroup 
label and reclassified based upon the subset of features that best describes its type of
event. 

Types were chosen such that members of a given type generally have many similar features. 
The set of features that is used to determine the specific event that occurred in stage two
is the set which varies most among the members.


\section{Results}
\label{sec:results}
In order to evaluate the performance of our system we examined the percentage of frames 
correctly classified with our segmentation and with perfect segmentation. The perfect
segmentation and ground truth label for each frame were determined using the annotation
to the development data.

The percentage of frames correct given perfect segmentation gives a sense of how well the
classification stage of our system performs. Currently, given perfect segmentation,
our system correctly classifies 48\% of frames. 

The percentage of frames correct with segmentation determined by our system when compared to the other metric
gives a good sense of how segmentation effects classification. With our segmentation, the 
system correctly classified 25\% of frames.

The confusion matricies of classifier stages of the perfectly segmented events, as well as the events segmented using our algorithm can be seen in Figure \ref{fig:confmat_perfect} and Figure \ref{fig:confmat_seg} respectively. (We need to make the figure bigger...)

\begin{figure}[h]
  \centering
  \centerline{\includegraphics[width=\columnwidth]{confmatrix1-eps-converted-to.pdf}}
  \caption{Example of a figure with experimental results.}
  \label{fig:confmat_perfect}
\end{figure}

\begin{figure}[h]
  \centering
  \centerline{\includegraphics[width=\columnwidth]{confmatrix2-eps-converted-to.pdf}}
  \caption{Example of a figure with experimental results.}
  \label{fig:confmat_seg}
\end{figure}


\section{Conclusion}
\label{sec:conclusion}
In a k-folds sense, our classification system performed well. This implies that our system
is not deficient in features that accurately classify events in this challenge. 

The major challenges that hindered the performance of our classification were noise in the 
development set and the problem of segmenting the input file into individual events. The 
noise also posed a major challenge in segmenting the file.



% Below is an example of how to insert images. 
% -------------------------------------------------------------------------

\section{REFERENCES}
\label{sec:ref}

List and number all bibliographical references at the end 
of the paper. The references should be numbered in order 
of appearance in the document. When referring to them in 
the text, type the corresponding reference number in 
square brackets as shown at the end of this sentence 
\cite{cJones2003}, \cite{aSmith2000}. For \LaTeX\ users, 
the use of the Bib\TeX\ style file IEEEtran.bst is 
recommended, which is included in the \LaTeX\ paper 
kit available from the workshop website \cite{aaspweb}.

\section{ACKNOWLEDGMENT}
\label{sec:ack}

Available Citations:
\cite{segmentFeature}  And 
\cite{silenceRemove} and
 \cite{prt2011} and 
\cite{discrimPRT} and
 \cite{dnn} and
 \cite{audioFeatures} and 
\cite{compModel}
\cite{rwEventD}
We would like to thank to Professor Sam Keene for his support and guidance. 
We would also like to thank New Folder Consulting for allowing us to use PRT.


% -------------------------------------------------------------------------
% Either list references using the bibliography style file IEEEtran.bst
%\bibliographystyle{IEEEtran}
%\bibliography{refsaasp}

% or list them by yourself
 \begin{thebibliography}{8}
\bibitem{silenceRemove}
	Theodoros Giannakopoulos, ``A method for silence removal and segmentation of speech signals, implemented in Matlab",\emph{ Department of Informatics and Telecommunications University of Athens, Greece|}, 2009. Software available at \url{https://www.mathworks.com/matlabcentral/fileexchange/28826-silence-removal-in-speech-signals}.

\bibitem{prt2011}
 Peter Torrione and Sam Keene and Kenneth Morton, ``{PRT}: The Pattern Recognition Toolbox for {MATLAB}", 2011. Software available at \url{http://newfolderconsulting.com/prt}.

\bibitem{segmentFeature}
 	Gordern Wichern, Jiachen Xue, Harvey Thornburg, Brandon Mechtley, and Andrewwas Spanias, ``Segmentation, Indexing, and Retrieval for Environmental and Natural Sounds" in \emph{"IEEE Trans. Signal Processing}, vol. 18, pp 688--707, March 2010.

\bibitem{audioFeatures}
 Theodoros Giannakopoulo, ``Some Basic Audio Features", Apr. 2008. Software available at \url{http://www.mathworks.com/matlabcentral/fileexchange/19236-some-basic-audio-features}

\bibitem{compModel}
Pampalk, Elias, ``Computational Models of Music Similarity and Their Application in Music Information Retrieval."\emph{Vienna University of Technology}, Mar. 2006

\bibitem{rwEventD}
Xiaodan Zhuang and Xi Zhou and Mark A and Hasegawa-Johnson and Thomas S. Huang, ``Real-world acoustic event detection", \emph{Pattern Recognition Letters} vol. 31, pp 1543--1551, Feb 2010

 \bibitem{dnn}
Geoffrey Hinton and Li Deng and Dong Yu and George Dahl and Abdel-rahman Mohamed and Navdeep Jaitly and Vincent Vanhoucke and Patrick Nguyen and Tara Sainath and Brian Kingsbury,  ``Deep Neural Networks for Acoustic Modeling in Speech Recognition,'' in \emph{IEEE Signal Processing Magazine}, pp. 82--97, Nov. 2012
 
 \bibitem{discrimPRT}
Xiaodong and Wu Chou. ``Discriminative Learning in Sequential Pattern Recognition,''  in \emph{IEEE Signal Processing Magazine}, pp. 14--36, Sept. 2008.
 
 \end{thebibliography}


\end{sloppy}
\end{document}
